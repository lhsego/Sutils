\documentclass{article}
\usepackage{graphicx, color}
\newcommand{\hlnumber}[1]{\textcolor[rgb]{0,0,0}{#1}}%
\newcommand{\hlfunctioncall}[1]{\textcolor[rgb]{.5,0,.33}{\textbf{#1}}}%
\newcommand{\hlstring}[1]{\textcolor[rgb]{.6,.6,1}{#1}}%
\newcommand{\hlkeyword}[1]{\textbf{#1}}%
\newcommand{\hlargument}[1]{\textcolor[rgb]{.69,.25,.02}{#1}}%
\newcommand{\hlcomment}[1]{\textcolor[rgb]{.18,.6,.34}{#1}}%
\newcommand{\hlroxygencomment}[1]{\textcolor[rgb]{.44,.48,.7}{#1}}%
\newcommand{\hlformalargs}[1]{\hlargument{#1}}%
\newcommand{\hleqformalargs}[1]{\hlargument{#1}}%
\newcommand{\hlassignement}[1]{\textbf{#1}}%
\newcommand{\hlpackage}[1]{\textcolor[rgb]{.59,.71,.145}{#1}}%
\newcommand{\hlslot}[1]{\textit{#1}}%
\newcommand{\hlsymbol}[1]{#1}%
\newcommand{\hlprompt}[1]{\textcolor[rgb]{.5,.5,.5}{#1}}%

\usepackage{color}%
 
\newsavebox{\hlnormalsizeboxclosebrace}%
\newsavebox{\hlnormalsizeboxopenbrace}%
\newsavebox{\hlnormalsizeboxbackslash}%
\newsavebox{\hlnormalsizeboxlessthan}%
\newsavebox{\hlnormalsizeboxgreaterthan}%
\newsavebox{\hlnormalsizeboxdollar}%
\newsavebox{\hlnormalsizeboxunderscore}%
\newsavebox{\hlnormalsizeboxand}%
\newsavebox{\hlnormalsizeboxhash}%
\newsavebox{\hlnormalsizeboxat}%
\newsavebox{\hlnormalsizeboxpercent}% 
\newsavebox{\hlnormalsizeboxhat}%
\newsavebox{\hlnormalsizeboxsinglequote}%
\newsavebox{\hlnormalsizeboxbacktick}%

\setbox\hlnormalsizeboxopenbrace=\hbox{\begin{normalsize}\verb.{.\end{normalsize}}%
\setbox\hlnormalsizeboxclosebrace=\hbox{\begin{normalsize}\verb.}.\end{normalsize}}%
\setbox\hlnormalsizeboxlessthan=\hbox{\begin{normalsize}\verb.<.\end{normalsize}}%
\setbox\hlnormalsizeboxdollar=\hbox{\begin{normalsize}\verb.$.\end{normalsize}}%
\setbox\hlnormalsizeboxunderscore=\hbox{\begin{normalsize}\verb._.\end{normalsize}}%
\setbox\hlnormalsizeboxand=\hbox{\begin{normalsize}\verb.&.\end{normalsize}}%
\setbox\hlnormalsizeboxhash=\hbox{\begin{normalsize}\verb.#.\end{normalsize}}%
\setbox\hlnormalsizeboxat=\hbox{\begin{normalsize}\verb.@.\end{normalsize}}%
\setbox\hlnormalsizeboxbackslash=\hbox{\begin{normalsize}\verb.\.\end{normalsize}}%
\setbox\hlnormalsizeboxgreaterthan=\hbox{\begin{normalsize}\verb.>.\end{normalsize}}%
\setbox\hlnormalsizeboxpercent=\hbox{\begin{normalsize}\verb.%.\end{normalsize}}%
\setbox\hlnormalsizeboxhat=\hbox{\begin{normalsize}\verb.^.\end{normalsize}}%
\setbox\hlnormalsizeboxsinglequote=\hbox{\begin{normalsize}\verb.'.\end{normalsize}}%
\setbox\hlnormalsizeboxbacktick=\hbox{\begin{normalsize}\verb.`.\end{normalsize}}%
\setbox\hlnormalsizeboxhat=\hbox{\begin{normalsize}\verb.^.\end{normalsize}}%



\newsavebox{\hltinyboxclosebrace}%
\newsavebox{\hltinyboxopenbrace}%
\newsavebox{\hltinyboxbackslash}%
\newsavebox{\hltinyboxlessthan}%
\newsavebox{\hltinyboxgreaterthan}%
\newsavebox{\hltinyboxdollar}%
\newsavebox{\hltinyboxunderscore}%
\newsavebox{\hltinyboxand}%
\newsavebox{\hltinyboxhash}%
\newsavebox{\hltinyboxat}%
\newsavebox{\hltinyboxpercent}% 
\newsavebox{\hltinyboxhat}%
\newsavebox{\hltinyboxsinglequote}%
\newsavebox{\hltinyboxbacktick}%

\setbox\hltinyboxopenbrace=\hbox{\begin{tiny}\verb.{.\end{tiny}}%
\setbox\hltinyboxclosebrace=\hbox{\begin{tiny}\verb.}.\end{tiny}}%
\setbox\hltinyboxlessthan=\hbox{\begin{tiny}\verb.<.\end{tiny}}%
\setbox\hltinyboxdollar=\hbox{\begin{tiny}\verb.$.\end{tiny}}%
\setbox\hltinyboxunderscore=\hbox{\begin{tiny}\verb._.\end{tiny}}%
\setbox\hltinyboxand=\hbox{\begin{tiny}\verb.&.\end{tiny}}%
\setbox\hltinyboxhash=\hbox{\begin{tiny}\verb.#.\end{tiny}}%
\setbox\hltinyboxat=\hbox{\begin{tiny}\verb.@.\end{tiny}}%
\setbox\hltinyboxbackslash=\hbox{\begin{tiny}\verb.\.\end{tiny}}%
\setbox\hltinyboxgreaterthan=\hbox{\begin{tiny}\verb.>.\end{tiny}}%
\setbox\hltinyboxpercent=\hbox{\begin{tiny}\verb.%.\end{tiny}}%
\setbox\hltinyboxhat=\hbox{\begin{tiny}\verb.^.\end{tiny}}%
\setbox\hltinyboxsinglequote=\hbox{\begin{tiny}\verb.'.\end{tiny}}%
\setbox\hltinyboxbacktick=\hbox{\begin{tiny}\verb.`.\end{tiny}}%
\setbox\hltinyboxhat=\hbox{\begin{tiny}\verb.^.\end{tiny}}%



\newsavebox{\hlscriptsizeboxclosebrace}%
\newsavebox{\hlscriptsizeboxopenbrace}%
\newsavebox{\hlscriptsizeboxbackslash}%
\newsavebox{\hlscriptsizeboxlessthan}%
\newsavebox{\hlscriptsizeboxgreaterthan}%
\newsavebox{\hlscriptsizeboxdollar}%
\newsavebox{\hlscriptsizeboxunderscore}%
\newsavebox{\hlscriptsizeboxand}%
\newsavebox{\hlscriptsizeboxhash}%
\newsavebox{\hlscriptsizeboxat}%
\newsavebox{\hlscriptsizeboxpercent}% 
\newsavebox{\hlscriptsizeboxhat}%
\newsavebox{\hlscriptsizeboxsinglequote}%
\newsavebox{\hlscriptsizeboxbacktick}%

\setbox\hlscriptsizeboxopenbrace=\hbox{\begin{scriptsize}\verb.{.\end{scriptsize}}%
\setbox\hlscriptsizeboxclosebrace=\hbox{\begin{scriptsize}\verb.}.\end{scriptsize}}%
\setbox\hlscriptsizeboxlessthan=\hbox{\begin{scriptsize}\verb.<.\end{scriptsize}}%
\setbox\hlscriptsizeboxdollar=\hbox{\begin{scriptsize}\verb.$.\end{scriptsize}}%
\setbox\hlscriptsizeboxunderscore=\hbox{\begin{scriptsize}\verb._.\end{scriptsize}}%
\setbox\hlscriptsizeboxand=\hbox{\begin{scriptsize}\verb.&.\end{scriptsize}}%
\setbox\hlscriptsizeboxhash=\hbox{\begin{scriptsize}\verb.#.\end{scriptsize}}%
\setbox\hlscriptsizeboxat=\hbox{\begin{scriptsize}\verb.@.\end{scriptsize}}%
\setbox\hlscriptsizeboxbackslash=\hbox{\begin{scriptsize}\verb.\.\end{scriptsize}}%
\setbox\hlscriptsizeboxgreaterthan=\hbox{\begin{scriptsize}\verb.>.\end{scriptsize}}%
\setbox\hlscriptsizeboxpercent=\hbox{\begin{scriptsize}\verb.%.\end{scriptsize}}%
\setbox\hlscriptsizeboxhat=\hbox{\begin{scriptsize}\verb.^.\end{scriptsize}}%
\setbox\hlscriptsizeboxsinglequote=\hbox{\begin{scriptsize}\verb.'.\end{scriptsize}}%
\setbox\hlscriptsizeboxbacktick=\hbox{\begin{scriptsize}\verb.`.\end{scriptsize}}%
\setbox\hlscriptsizeboxhat=\hbox{\begin{scriptsize}\verb.^.\end{scriptsize}}%



\newsavebox{\hlfootnotesizeboxclosebrace}%
\newsavebox{\hlfootnotesizeboxopenbrace}%
\newsavebox{\hlfootnotesizeboxbackslash}%
\newsavebox{\hlfootnotesizeboxlessthan}%
\newsavebox{\hlfootnotesizeboxgreaterthan}%
\newsavebox{\hlfootnotesizeboxdollar}%
\newsavebox{\hlfootnotesizeboxunderscore}%
\newsavebox{\hlfootnotesizeboxand}%
\newsavebox{\hlfootnotesizeboxhash}%
\newsavebox{\hlfootnotesizeboxat}%
\newsavebox{\hlfootnotesizeboxpercent}% 
\newsavebox{\hlfootnotesizeboxhat}%
\newsavebox{\hlfootnotesizeboxsinglequote}%
\newsavebox{\hlfootnotesizeboxbacktick}%

\setbox\hlfootnotesizeboxopenbrace=\hbox{\begin{footnotesize}\verb.{.\end{footnotesize}}%
\setbox\hlfootnotesizeboxclosebrace=\hbox{\begin{footnotesize}\verb.}.\end{footnotesize}}%
\setbox\hlfootnotesizeboxlessthan=\hbox{\begin{footnotesize}\verb.<.\end{footnotesize}}%
\setbox\hlfootnotesizeboxdollar=\hbox{\begin{footnotesize}\verb.$.\end{footnotesize}}%
\setbox\hlfootnotesizeboxunderscore=\hbox{\begin{footnotesize}\verb._.\end{footnotesize}}%
\setbox\hlfootnotesizeboxand=\hbox{\begin{footnotesize}\verb.&.\end{footnotesize}}%
\setbox\hlfootnotesizeboxhash=\hbox{\begin{footnotesize}\verb.#.\end{footnotesize}}%
\setbox\hlfootnotesizeboxat=\hbox{\begin{footnotesize}\verb.@.\end{footnotesize}}%
\setbox\hlfootnotesizeboxbackslash=\hbox{\begin{footnotesize}\verb.\.\end{footnotesize}}%
\setbox\hlfootnotesizeboxgreaterthan=\hbox{\begin{footnotesize}\verb.>.\end{footnotesize}}%
\setbox\hlfootnotesizeboxpercent=\hbox{\begin{footnotesize}\verb.%.\end{footnotesize}}%
\setbox\hlfootnotesizeboxhat=\hbox{\begin{footnotesize}\verb.^.\end{footnotesize}}%
\setbox\hlfootnotesizeboxsinglequote=\hbox{\begin{footnotesize}\verb.'.\end{footnotesize}}%
\setbox\hlfootnotesizeboxbacktick=\hbox{\begin{footnotesize}\verb.`.\end{footnotesize}}%
\setbox\hlfootnotesizeboxhat=\hbox{\begin{footnotesize}\verb.^.\end{footnotesize}}%



\newsavebox{\hlsmallboxclosebrace}%
\newsavebox{\hlsmallboxopenbrace}%
\newsavebox{\hlsmallboxbackslash}%
\newsavebox{\hlsmallboxlessthan}%
\newsavebox{\hlsmallboxgreaterthan}%
\newsavebox{\hlsmallboxdollar}%
\newsavebox{\hlsmallboxunderscore}%
\newsavebox{\hlsmallboxand}%
\newsavebox{\hlsmallboxhash}%
\newsavebox{\hlsmallboxat}%
\newsavebox{\hlsmallboxpercent}% 
\newsavebox{\hlsmallboxhat}%
\newsavebox{\hlsmallboxsinglequote}%
\newsavebox{\hlsmallboxbacktick}%

\setbox\hlsmallboxopenbrace=\hbox{\begin{small}\verb.{.\end{small}}%
\setbox\hlsmallboxclosebrace=\hbox{\begin{small}\verb.}.\end{small}}%
\setbox\hlsmallboxlessthan=\hbox{\begin{small}\verb.<.\end{small}}%
\setbox\hlsmallboxdollar=\hbox{\begin{small}\verb.$.\end{small}}%
\setbox\hlsmallboxunderscore=\hbox{\begin{small}\verb._.\end{small}}%
\setbox\hlsmallboxand=\hbox{\begin{small}\verb.&.\end{small}}%
\setbox\hlsmallboxhash=\hbox{\begin{small}\verb.#.\end{small}}%
\setbox\hlsmallboxat=\hbox{\begin{small}\verb.@.\end{small}}%
\setbox\hlsmallboxbackslash=\hbox{\begin{small}\verb.\.\end{small}}%
\setbox\hlsmallboxgreaterthan=\hbox{\begin{small}\verb.>.\end{small}}%
\setbox\hlsmallboxpercent=\hbox{\begin{small}\verb.%.\end{small}}%
\setbox\hlsmallboxhat=\hbox{\begin{small}\verb.^.\end{small}}%
\setbox\hlsmallboxsinglequote=\hbox{\begin{small}\verb.'.\end{small}}%
\setbox\hlsmallboxbacktick=\hbox{\begin{small}\verb.`.\end{small}}%
\setbox\hlsmallboxhat=\hbox{\begin{small}\verb.^.\end{small}}%



\newsavebox{\hllargeboxclosebrace}%
\newsavebox{\hllargeboxopenbrace}%
\newsavebox{\hllargeboxbackslash}%
\newsavebox{\hllargeboxlessthan}%
\newsavebox{\hllargeboxgreaterthan}%
\newsavebox{\hllargeboxdollar}%
\newsavebox{\hllargeboxunderscore}%
\newsavebox{\hllargeboxand}%
\newsavebox{\hllargeboxhash}%
\newsavebox{\hllargeboxat}%
\newsavebox{\hllargeboxpercent}% 
\newsavebox{\hllargeboxhat}%
\newsavebox{\hllargeboxsinglequote}%
\newsavebox{\hllargeboxbacktick}%

\setbox\hllargeboxopenbrace=\hbox{\begin{large}\verb.{.\end{large}}%
\setbox\hllargeboxclosebrace=\hbox{\begin{large}\verb.}.\end{large}}%
\setbox\hllargeboxlessthan=\hbox{\begin{large}\verb.<.\end{large}}%
\setbox\hllargeboxdollar=\hbox{\begin{large}\verb.$.\end{large}}%
\setbox\hllargeboxunderscore=\hbox{\begin{large}\verb._.\end{large}}%
\setbox\hllargeboxand=\hbox{\begin{large}\verb.&.\end{large}}%
\setbox\hllargeboxhash=\hbox{\begin{large}\verb.#.\end{large}}%
\setbox\hllargeboxat=\hbox{\begin{large}\verb.@.\end{large}}%
\setbox\hllargeboxbackslash=\hbox{\begin{large}\verb.\.\end{large}}%
\setbox\hllargeboxgreaterthan=\hbox{\begin{large}\verb.>.\end{large}}%
\setbox\hllargeboxpercent=\hbox{\begin{large}\verb.%.\end{large}}%
\setbox\hllargeboxhat=\hbox{\begin{large}\verb.^.\end{large}}%
\setbox\hllargeboxsinglequote=\hbox{\begin{large}\verb.'.\end{large}}%
\setbox\hllargeboxbacktick=\hbox{\begin{large}\verb.`.\end{large}}%
\setbox\hllargeboxhat=\hbox{\begin{large}\verb.^.\end{large}}%



\newsavebox{\hlLargeboxclosebrace}%
\newsavebox{\hlLargeboxopenbrace}%
\newsavebox{\hlLargeboxbackslash}%
\newsavebox{\hlLargeboxlessthan}%
\newsavebox{\hlLargeboxgreaterthan}%
\newsavebox{\hlLargeboxdollar}%
\newsavebox{\hlLargeboxunderscore}%
\newsavebox{\hlLargeboxand}%
\newsavebox{\hlLargeboxhash}%
\newsavebox{\hlLargeboxat}%
\newsavebox{\hlLargeboxpercent}% 
\newsavebox{\hlLargeboxhat}%
\newsavebox{\hlLargeboxsinglequote}%
\newsavebox{\hlLargeboxbacktick}%

\setbox\hlLargeboxopenbrace=\hbox{\begin{Large}\verb.{.\end{Large}}%
\setbox\hlLargeboxclosebrace=\hbox{\begin{Large}\verb.}.\end{Large}}%
\setbox\hlLargeboxlessthan=\hbox{\begin{Large}\verb.<.\end{Large}}%
\setbox\hlLargeboxdollar=\hbox{\begin{Large}\verb.$.\end{Large}}%
\setbox\hlLargeboxunderscore=\hbox{\begin{Large}\verb._.\end{Large}}%
\setbox\hlLargeboxand=\hbox{\begin{Large}\verb.&.\end{Large}}%
\setbox\hlLargeboxhash=\hbox{\begin{Large}\verb.#.\end{Large}}%
\setbox\hlLargeboxat=\hbox{\begin{Large}\verb.@.\end{Large}}%
\setbox\hlLargeboxbackslash=\hbox{\begin{Large}\verb.\.\end{Large}}%
\setbox\hlLargeboxgreaterthan=\hbox{\begin{Large}\verb.>.\end{Large}}%
\setbox\hlLargeboxpercent=\hbox{\begin{Large}\verb.%.\end{Large}}%
\setbox\hlLargeboxhat=\hbox{\begin{Large}\verb.^.\end{Large}}%
\setbox\hlLargeboxsinglequote=\hbox{\begin{Large}\verb.'.\end{Large}}%
\setbox\hlLargeboxbacktick=\hbox{\begin{Large}\verb.`.\end{Large}}%
\setbox\hlLargeboxhat=\hbox{\begin{Large}\verb.^.\end{Large}}%



\newsavebox{\hlLARGEboxclosebrace}%
\newsavebox{\hlLARGEboxopenbrace}%
\newsavebox{\hlLARGEboxbackslash}%
\newsavebox{\hlLARGEboxlessthan}%
\newsavebox{\hlLARGEboxgreaterthan}%
\newsavebox{\hlLARGEboxdollar}%
\newsavebox{\hlLARGEboxunderscore}%
\newsavebox{\hlLARGEboxand}%
\newsavebox{\hlLARGEboxhash}%
\newsavebox{\hlLARGEboxat}%
\newsavebox{\hlLARGEboxpercent}% 
\newsavebox{\hlLARGEboxhat}%
\newsavebox{\hlLARGEboxsinglequote}%
\newsavebox{\hlLARGEboxbacktick}%

\setbox\hlLARGEboxopenbrace=\hbox{\begin{LARGE}\verb.{.\end{LARGE}}%
\setbox\hlLARGEboxclosebrace=\hbox{\begin{LARGE}\verb.}.\end{LARGE}}%
\setbox\hlLARGEboxlessthan=\hbox{\begin{LARGE}\verb.<.\end{LARGE}}%
\setbox\hlLARGEboxdollar=\hbox{\begin{LARGE}\verb.$.\end{LARGE}}%
\setbox\hlLARGEboxunderscore=\hbox{\begin{LARGE}\verb._.\end{LARGE}}%
\setbox\hlLARGEboxand=\hbox{\begin{LARGE}\verb.&.\end{LARGE}}%
\setbox\hlLARGEboxhash=\hbox{\begin{LARGE}\verb.#.\end{LARGE}}%
\setbox\hlLARGEboxat=\hbox{\begin{LARGE}\verb.@.\end{LARGE}}%
\setbox\hlLARGEboxbackslash=\hbox{\begin{LARGE}\verb.\.\end{LARGE}}%
\setbox\hlLARGEboxgreaterthan=\hbox{\begin{LARGE}\verb.>.\end{LARGE}}%
\setbox\hlLARGEboxpercent=\hbox{\begin{LARGE}\verb.%.\end{LARGE}}%
\setbox\hlLARGEboxhat=\hbox{\begin{LARGE}\verb.^.\end{LARGE}}%
\setbox\hlLARGEboxsinglequote=\hbox{\begin{LARGE}\verb.'.\end{LARGE}}%
\setbox\hlLARGEboxbacktick=\hbox{\begin{LARGE}\verb.`.\end{LARGE}}%
\setbox\hlLARGEboxhat=\hbox{\begin{LARGE}\verb.^.\end{LARGE}}%



\newsavebox{\hlhugeboxclosebrace}%
\newsavebox{\hlhugeboxopenbrace}%
\newsavebox{\hlhugeboxbackslash}%
\newsavebox{\hlhugeboxlessthan}%
\newsavebox{\hlhugeboxgreaterthan}%
\newsavebox{\hlhugeboxdollar}%
\newsavebox{\hlhugeboxunderscore}%
\newsavebox{\hlhugeboxand}%
\newsavebox{\hlhugeboxhash}%
\newsavebox{\hlhugeboxat}%
\newsavebox{\hlhugeboxpercent}% 
\newsavebox{\hlhugeboxhat}%
\newsavebox{\hlhugeboxsinglequote}%
\newsavebox{\hlhugeboxbacktick}%

\setbox\hlhugeboxopenbrace=\hbox{\begin{huge}\verb.{.\end{huge}}%
\setbox\hlhugeboxclosebrace=\hbox{\begin{huge}\verb.}.\end{huge}}%
\setbox\hlhugeboxlessthan=\hbox{\begin{huge}\verb.<.\end{huge}}%
\setbox\hlhugeboxdollar=\hbox{\begin{huge}\verb.$.\end{huge}}%
\setbox\hlhugeboxunderscore=\hbox{\begin{huge}\verb._.\end{huge}}%
\setbox\hlhugeboxand=\hbox{\begin{huge}\verb.&.\end{huge}}%
\setbox\hlhugeboxhash=\hbox{\begin{huge}\verb.#.\end{huge}}%
\setbox\hlhugeboxat=\hbox{\begin{huge}\verb.@.\end{huge}}%
\setbox\hlhugeboxbackslash=\hbox{\begin{huge}\verb.\.\end{huge}}%
\setbox\hlhugeboxgreaterthan=\hbox{\begin{huge}\verb.>.\end{huge}}%
\setbox\hlhugeboxpercent=\hbox{\begin{huge}\verb.%.\end{huge}}%
\setbox\hlhugeboxhat=\hbox{\begin{huge}\verb.^.\end{huge}}%
\setbox\hlhugeboxsinglequote=\hbox{\begin{huge}\verb.'.\end{huge}}%
\setbox\hlhugeboxbacktick=\hbox{\begin{huge}\verb.`.\end{huge}}%
\setbox\hlhugeboxhat=\hbox{\begin{huge}\verb.^.\end{huge}}%



\newsavebox{\hlHugeboxclosebrace}%
\newsavebox{\hlHugeboxopenbrace}%
\newsavebox{\hlHugeboxbackslash}%
\newsavebox{\hlHugeboxlessthan}%
\newsavebox{\hlHugeboxgreaterthan}%
\newsavebox{\hlHugeboxdollar}%
\newsavebox{\hlHugeboxunderscore}%
\newsavebox{\hlHugeboxand}%
\newsavebox{\hlHugeboxhash}%
\newsavebox{\hlHugeboxat}%
\newsavebox{\hlHugeboxpercent}% 
\newsavebox{\hlHugeboxhat}%
\newsavebox{\hlHugeboxsinglequote}%
\newsavebox{\hlHugeboxbacktick}%

\setbox\hlHugeboxopenbrace=\hbox{\begin{Huge}\verb.{.\end{Huge}}%
\setbox\hlHugeboxclosebrace=\hbox{\begin{Huge}\verb.}.\end{Huge}}%
\setbox\hlHugeboxlessthan=\hbox{\begin{Huge}\verb.<.\end{Huge}}%
\setbox\hlHugeboxdollar=\hbox{\begin{Huge}\verb.$.\end{Huge}}%
\setbox\hlHugeboxunderscore=\hbox{\begin{Huge}\verb._.\end{Huge}}%
\setbox\hlHugeboxand=\hbox{\begin{Huge}\verb.&.\end{Huge}}%
\setbox\hlHugeboxhash=\hbox{\begin{Huge}\verb.#.\end{Huge}}%
\setbox\hlHugeboxat=\hbox{\begin{Huge}\verb.@.\end{Huge}}%
\setbox\hlHugeboxbackslash=\hbox{\begin{Huge}\verb.\.\end{Huge}}%
\setbox\hlHugeboxgreaterthan=\hbox{\begin{Huge}\verb.>.\end{Huge}}%
\setbox\hlHugeboxpercent=\hbox{\begin{Huge}\verb.%.\end{Huge}}%
\setbox\hlHugeboxhat=\hbox{\begin{Huge}\verb.^.\end{Huge}}%
\setbox\hlHugeboxsinglequote=\hbox{\begin{Huge}\verb.'.\end{Huge}}%
\setbox\hlHugeboxbacktick=\hbox{\begin{Huge}\verb.`.\end{Huge}}%
\setbox\hlHugeboxhat=\hbox{\begin{Huge}\verb.^.\end{Huge}}%
 

\def\urltilda{\kern -.15em\lower .7ex\hbox{\~{}}\kern .04em}%

\newcommand{\hlstd}[1]{\textcolor[rgb]{0,0,0}{#1}}%
\newcommand{\hlnum}[1]{\textcolor[rgb]{0.16,0.16,1}{#1}}
\newcommand{\hlesc}[1]{\textcolor[rgb]{1,0,1}{#1}}
\newcommand{\hlstr}[1]{\textcolor[rgb]{1,0,0}{#1}}
\newcommand{\hldstr}[1]{\textcolor[rgb]{0.51,0.51,0}{#1}}
\newcommand{\hlslc}[1]{\textcolor[rgb]{0.51,0.51,0.51}{\it{#1}}}
\newcommand{\hlcom}[1]{\textcolor[rgb]{0.51,0.51,0.51}{\it{#1}}}
\newcommand{\hldir}[1]{\textcolor[rgb]{0,0.51,0}{#1}}
\newcommand{\hlsym}[1]{\textcolor[rgb]{0,0,0}{#1}}
\newcommand{\hlline}[1]{\textcolor[rgb]{0.33,0.33,0.33}{#1}}
\newcommand{\hlkwa}[1]{\textcolor[rgb]{0,0,0}{\bf{#1}}}
\newcommand{\hlkwb}[1]{\textcolor[rgb]{0.51,0,0}{#1}}
\newcommand{\hlkwc}[1]{\textcolor[rgb]{0,0,0}{\bf{#1}}}
\newcommand{\hlkwd}[1]{\textcolor[rgb]{0,0,0.51}{#1}}

\definecolor{fgcolor}{rgb}{0,0,0}
\usepackage{framed}
\makeatletter
\newenvironment{kframe}{%
 \def\FrameCommand##1{\hskip\@totalleftmargin \hskip-\fboxsep
 \colorbox{shadecolor}{##1}\hskip-\fboxsep
     % There is no \@totalrightmargin, so:
     \hskip-\linewidth \hskip-\@totalleftmargin \hskip\columnwidth}%
 \MakeFramed {\advance\hsize-\width
   \@totalleftmargin\z@ \linewidth\hsize
   \@setminipage}}%
 {\par\unskip\endMakeFramed}
\makeatother

\definecolor{shadecolor}{rgb}{.97, .97, .97}
\newenvironment{knitrout}{}{} % an empty environment to be redefined in TeX

\newcommand{\SweaveOpts}[1]{}  % do not interfere with LaTeX
\newcommand{\SweaveInput}[1]{} % because they are not real TeX commands
\newcommand{\Sexpr}[1]{}       % will only be parsed by R

\usepackage{url}

\title{{\tt SQM}: An R Package for Signature Quality Metrics\\The Accuracy Component}
\author{John A. Ramey}

\begin{document}

\maketitle

\section{Introduction}

The {\tt SQM} package contains the R code for the Signature Quality Metrics project
under the Signature Discovery Initiative (SDI) at the Pacific Northwest National
Laboratory (PNNL). The purpose of the SQM project is to provide subject-matter
experts (SMEs) with a set of tools to assess the quality of a set of signatures
in terms of accuracy, cost, and utility. For each of these components, we
have provided easy-to-use, accessible functions to measure the quality of
inputted signatures. In this document, we discuss each of the SQM components
and demonstrate their usage with realistic examples\footnote{Familarity with R
is recommended to fully comprehend our provided examples.}.

The {\tt Accuracy} component of {\tt SQM} enables an SME to explore the efficacy
of a set of candidate signatures to determine which of these signatures are
nearly optimal as defined by a variety of available accuracy measures. Paired
with other SQM components, such as {\tt Cost}, the {\tt Accuracy} component
equips an SME to identify cost-effective signatures that attain excellent
accuracy.

To motivate the {\tt Accuracy} component, we use a widely-studied, public domain
data set with six signatures that we construct below. For demonstration purposes,
we use the UCI Machine Learning Repository's Satellite data set\footnote{For more
details about the Satellite data set, see
\url{http://archive.ics.uci.edu/ml/datasets/Statlog+(Landsat+Satellite)}.},
which is available in the {\tt mlbench} R package. 

\subsection{Signature Construction}

The {\tt SQM} package requires that candidate signatures are previously
constructed. To demonstrate an approach this process, we first construct six
candidate signatures from the Satellite data set using three machine learning
algorithms, known as \emph{classifiers}. The three classifiers are applied to the
Satellite data set to obtain the first three signatures. The remaining signatures
are constructed by preprocessing the Satellite data set and then applying the
same three classifiers.

We begin the signature construction by loading the necessary R packages and the
Satellite data set.

\begin{knitrout}
\definecolor{shadecolor}{rgb}{0.969, 0.969, 0.969}\color{fgcolor}\begin{kframe}
\begin{flushleft}
\ttfamily\noindent
\hlfunctioncall{library}\hlkeyword{(}\hlstring{"{}caret"{}}\hlkeyword{)}\hspace*{\fill}\\
\hlstd{}\hlfunctioncall{library}\hlkeyword{(}\hlstring{"{}SQM"{}}\hlkeyword{)}\hspace*{\fill}\\
\hlstd{}\hlfunctioncall{library}\hlkeyword{(}\hlstring{"{}mlbench"{}}\hlkeyword{)}\hspace*{\fill}\\
\hlstd{}\hlfunctioncall{library}\hlkeyword{(}\hlstring{"{}reshape2"{}}\hlkeyword{)}\hspace*{\fill}\\
\hlstd{}\hlfunctioncall{data}\hlkeyword{(}\hlstring{"{}Satellite"{}}\hlkeyword{)}\mbox{}
\normalfont
\end{flushleft}
\end{kframe}
\end{knitrout}


The Satellite data is provided in a {\tt data.frame}. For clarity, we explicitly
assign the variables (i.e covariates/feature vectors) into the usual {\tt X}
matrix. Also, we note that the true classes of each observation are given, and we
assign these to the usual {\tt Y} vector.

\begin{knitrout}
\definecolor{shadecolor}{rgb}{0.969, 0.969, 0.969}\color{fgcolor}\begin{kframe}
\begin{flushleft}
\ttfamily\noindent
\hlsymbol{satelliteX}{\ }\hlassignement{\usebox{\hlnormalsizeboxlessthan}-}{\ }\hlfunctioncall{subset}\hlkeyword{(}\hlsymbol{Satellite}\hlkeyword{,}{\ }\hlargument{select}{\ }\hlargument{=}{\ }\hlkeyword{-}\hlsymbol{classes}\hlkeyword{)}\hspace*{\fill}\\
\hlstd{}\hlsymbol{satelliteY}{\ }\hlassignement{\usebox{\hlnormalsizeboxlessthan}-}{\ }\hlsymbol{Satellite}\hlkeyword{\usebox{\hlnormalsizeboxdollar}}\hlsymbol{classes}\mbox{}
\normalfont
\end{flushleft}
\end{kframe}
\end{knitrout}


As provided, the class labels in the Satellite data set have white space, which
can lead to annoying warnings. To stifle these warnings, we replace the
whitespace in the class labels with periods.

\begin{knitrout}
\definecolor{shadecolor}{rgb}{0.969, 0.969, 0.969}\color{fgcolor}\begin{kframe}
\begin{flushleft}
\ttfamily\noindent
\hlfunctioncall{levels}\hlkeyword{(}\hlsymbol{satelliteY}\hlkeyword{)}{\ }\hlassignement{\usebox{\hlnormalsizeboxlessthan}-}{\ }\hlfunctioncall{gsub}\hlkeyword{(}\hlstring{"{}{\ }"{}}\hlkeyword{,}{\ }\hlstring{"{}."{}}\hlkeyword{,}{\ }\hlfunctioncall{levels}\hlkeyword{(}\hlsymbol{satelliteY}\hlkeyword{)}\hlkeyword{)}\mbox{}
\normalfont
\end{flushleft}
\end{kframe}
\end{knitrout}


Next, we randomly partition the Satellite data set into a training and a test
data set, where the training data set has 1/4 of the original observations and
the test data set contains the remaining observations. To partition the data set,
we use the {\tt createDataPartition} function from the {\tt caret} package after
setting a fixed seed for the random number generator to achieve reproducible
results.

\begin{knitrout}
\definecolor{shadecolor}{rgb}{0.969, 0.969, 0.969}\color{fgcolor}\begin{kframe}
\begin{flushleft}
\ttfamily\noindent
\hlfunctioncall{set.seed}\hlkeyword{(}\hlnumber{42}\hlkeyword{)}\hspace*{\fill}\\
\hlstd{}\hlsymbol{inTrain}{\ }\hlassignement{\usebox{\hlnormalsizeboxlessthan}-}{\ }\hlfunctioncall{createDataPartition}\hlkeyword{(}\hlsymbol{satelliteY}\hlkeyword{,}{\ }\hlargument{p}{\ }\hlargument{=}{\ }\hlnumber{1}\hlkeyword{/}\hlnumber{4}\hlkeyword{,}{\ }\hlargument{list}{\ }\hlargument{=}{\ }\hlnumber{FALSE}\hlkeyword{,}\hspace*{\fill}\\
\hlstd{}{\ }{\ }{\ }{\ }\hlargument{times}{\ }\hlargument{=}{\ }\hlnumber{1}\hlkeyword{)}\hspace*{\fill}\\
\hlstd{}\hlsymbol{trainX}{\ }\hlassignement{\usebox{\hlnormalsizeboxlessthan}-}{\ }\hlsymbol{satelliteX}\hlkeyword{[}\hlsymbol{inTrain}\hlkeyword{,}{\ }\hlkeyword{]}\hspace*{\fill}\\
\hlstd{}\hlsymbol{testX}{\ }\hlassignement{\usebox{\hlnormalsizeboxlessthan}-}{\ }\hlsymbol{satelliteX}\hlkeyword{[}\hlkeyword{-}\hlsymbol{inTrain}\hlkeyword{,}{\ }\hlkeyword{]}\hspace*{\fill}\\
\hlstd{}\hlsymbol{trainY}{\ }\hlassignement{\usebox{\hlnormalsizeboxlessthan}-}{\ }\hlsymbol{satelliteY}\hlkeyword{[}\hlsymbol{inTrain}\hlkeyword{]}\hspace*{\fill}\\
\hlstd{}\hlsymbol{testY}{\ }\hlassignement{\usebox{\hlnormalsizeboxlessthan}-}{\ }\hlsymbol{satelliteY}\hlkeyword{[}\hlkeyword{-}\hlsymbol{inTrain}\hlkeyword{]}\mbox{}
\normalfont
\end{flushleft}
\end{kframe}
\end{knitrout}


The class labels for the observations in the data set are considered unknown from
the perspective of the signatures. Thus, the class labels in {\tt testY} are
considered as the unknown, ground truth. Below, we train three classifiers on the
training data set and classify the test observations in hopes that we correctly
identify their true classes, which are given in testY. For clarity, we store the
test classes as {\tt truthClass}. Each classifier should be viewed as a different
signature, and their test classifications/predictions are the {\tt predictedClass}
in the {\tt SQM} package.

\begin{knitrout}
\definecolor{shadecolor}{rgb}{0.969, 0.969, 0.969}\color{fgcolor}\begin{kframe}
\begin{flushleft}
\ttfamily\noindent
\hlsymbol{truthClass}{\ }\hlassignement{\usebox{\hlnormalsizeboxlessthan}-}{\ }\hlsymbol{testY}\mbox{}
\normalfont
\end{flushleft}
\end{kframe}
\end{knitrout}


We train each of the given classifiers with two training data sets.
The first data set is the unaltered training data set from above.
To construct the second data set, we keep only the features (variables) with
a pairwise correlation above 0.95 with another feature. The choice of 0.95 is
arbitrary. This feature selection approach is useful for demonstration but should
be viewed as naive in practice.

\begin{knitrout}
\definecolor{shadecolor}{rgb}{0.969, 0.969, 0.969}\color{fgcolor}\begin{kframe}
\begin{flushleft}
\ttfamily\noindent
\hlsymbol{keptFeatures}{\ }\hlassignement{\usebox{\hlnormalsizeboxlessthan}-}{\ }\hlfunctioncall{findCorrelation}\hlkeyword{(}\hlfunctioncall{cor}\hlkeyword{(}\hlsymbol{trainX}\hlkeyword{)}\hlkeyword{,}{\ }\hlargument{cutoff}{\ }\hlargument{=}{\ }\hlnumber{0.95}\hlkeyword{)}\hspace*{\fill}\\
\hlstd{}\hlsymbol{trainX2}{\ }\hlassignement{\usebox{\hlnormalsizeboxlessthan}-}{\ }\hlsymbol{trainX}\hlkeyword{[}\hlkeyword{,}{\ }\hlsymbol{keptFeatures}\hlkeyword{]}\hspace*{\fill}\\
\hlstd{}\hlsymbol{testX2}{\ }\hlassignement{\usebox{\hlnormalsizeboxlessthan}-}{\ }\hlsymbol{testX}\hlkeyword{[}\hlkeyword{,}{\ }\hlsymbol{keptFeatures}\hlkeyword{]}\mbox{}
\normalfont
\end{flushleft}
\end{kframe}
\end{knitrout}


To construct each signature, we apply 10-fold cross-validation with the help of
the {\tt trainControl} function given by the {\tt caret} package.
\begin{knitrout}
\definecolor{shadecolor}{rgb}{0.969, 0.969, 0.969}\color{fgcolor}\begin{kframe}
\begin{flushleft}
\ttfamily\noindent
\hlsymbol{fitControl}{\ }\hlassignement{\usebox{\hlnormalsizeboxlessthan}-}{\ }\hlfunctioncall{trainControl}\hlkeyword{(}\hlargument{method}{\ }\hlargument{=}{\ }\hlstring{"{}cv"{}}\hlkeyword{,}{\ }\hlargument{number}{\ }\hlargument{=}{\ }\hlnumber{10}\hlkeyword{)}\mbox{}
\normalfont
\end{flushleft}
\end{kframe}
\end{knitrout}


As we have discussed above, we utilize three classifiers in order to construct
the six signatures. These classifiers that we use are:

\begin{enumerate}
  \item Linear Discriminant Analysis
  \item $k$-Nearest Neighbors
  \item Support Vector Machine with Radial Basis Functions
\end{enumerate}

We train each classifier twice: first on the unaltered training data set and
then on the preprocessed data set. We use the {\tt train} function from the
{\tt caret} package to train the classifiers.

\begin{knitrout}
\definecolor{shadecolor}{rgb}{0.969, 0.969, 0.969}\color{fgcolor}\begin{kframe}
\begin{flushleft}
\ttfamily\noindent
\hlsymbol{ldaFit1}{\ }\hlassignement{\usebox{\hlnormalsizeboxlessthan}-}{\ }\hlfunctioncall{train}\hlkeyword{(}\hlsymbol{trainX}\hlkeyword{,}{\ }\hlsymbol{trainY}\hlkeyword{,}{\ }\hlargument{method}{\ }\hlargument{=}{\ }\hlstring{"{}lda"{}}\hlkeyword{,}{\ }\hlargument{trControl}{\ }\hlargument{=}{\ }\hlsymbol{fitControl}\hlkeyword{)}\hspace*{\fill}\\
\hlstd{}\hlsymbol{knnFit1}{\ }\hlassignement{\usebox{\hlnormalsizeboxlessthan}-}{\ }\hlfunctioncall{train}\hlkeyword{(}\hlsymbol{trainX}\hlkeyword{,}{\ }\hlsymbol{trainY}\hlkeyword{,}{\ }\hlargument{method}{\ }\hlargument{=}{\ }\hlstring{"{}knn"{}}\hlkeyword{,}{\ }\hlargument{trControl}{\ }\hlargument{=}{\ }\hlsymbol{fitControl}\hlkeyword{)}\hspace*{\fill}\\
\hlstd{}\hlsymbol{svmFit1}{\ }\hlassignement{\usebox{\hlnormalsizeboxlessthan}-}{\ }\hlfunctioncall{train}\hlkeyword{(}\hlsymbol{trainX}\hlkeyword{,}{\ }\hlsymbol{trainY}\hlkeyword{,}{\ }\hlargument{method}{\ }\hlargument{=}{\ }\hlstring{"{}svmRadial"{}}\hlkeyword{,}{\ }\hlargument{trControl}{\ }\hlargument{=}{\ }\hlsymbol{fitControl}\hlkeyword{)}\hspace*{\fill}\\
\hlstd{}\hspace*{\fill}\\
\hlstd{}\hlsymbol{ldaFit2}{\ }\hlassignement{\usebox{\hlnormalsizeboxlessthan}-}{\ }\hlfunctioncall{train}\hlkeyword{(}\hlsymbol{trainX2}\hlkeyword{,}{\ }\hlsymbol{trainY}\hlkeyword{,}{\ }\hlargument{method}{\ }\hlargument{=}{\ }\hlstring{"{}lda"{}}\hlkeyword{,}{\ }\hlargument{trControl}{\ }\hlargument{=}{\ }\hlsymbol{fitControl}\hlkeyword{)}\hspace*{\fill}\\
\hlstd{}\hlsymbol{knnFit2}{\ }\hlassignement{\usebox{\hlnormalsizeboxlessthan}-}{\ }\hlfunctioncall{train}\hlkeyword{(}\hlsymbol{trainX2}\hlkeyword{,}{\ }\hlsymbol{trainY}\hlkeyword{,}{\ }\hlargument{method}{\ }\hlargument{=}{\ }\hlstring{"{}knn"{}}\hlkeyword{,}{\ }\hlargument{trControl}{\ }\hlargument{=}{\ }\hlsymbol{fitControl}\hlkeyword{)}\hspace*{\fill}\\
\hlstd{}\hlsymbol{svmFit2}{\ }\hlassignement{\usebox{\hlnormalsizeboxlessthan}-}{\ }\hlfunctioncall{train}\hlkeyword{(}\hlsymbol{trainX2}\hlkeyword{,}{\ }\hlsymbol{trainY}\hlkeyword{,}{\ }\hlargument{method}{\ }\hlargument{=}{\ }\hlstring{"{}svmRadial"{}}\hlkeyword{,}{\ }\hlargument{trControl}{\ }\hlargument{=}{\ }\hlsymbol{fitControl}\hlkeyword{)}\mbox{}
\normalfont
\end{flushleft}
\end{kframe}
\end{knitrout}


Next, we formulate the fitted models as lists to easily classify the test data.

\begin{knitrout}
\definecolor{shadecolor}{rgb}{0.969, 0.969, 0.969}\color{fgcolor}\begin{kframe}
\begin{flushleft}
\ttfamily\noindent
\hlsymbol{fittedModels1}{\ }\hlassignement{\usebox{\hlnormalsizeboxlessthan}-}{\ }\hlfunctioncall{list}\hlkeyword{(}\hlargument{ldaFit1}{\ }\hlargument{=}{\ }\hlsymbol{ldaFit1}\hlkeyword{,}{\ }\hlargument{knnFit1}{\ }\hlargument{=}{\ }\hlsymbol{knnFit1}\hlkeyword{,}{\ }\hlargument{svmFit1}{\ }\hlargument{=}{\ }\hlsymbol{svmFit1}\hlkeyword{)}\hspace*{\fill}\\
\hlstd{}\hlsymbol{fittedModels2}{\ }\hlassignement{\usebox{\hlnormalsizeboxlessthan}-}{\ }\hlfunctioncall{list}\hlkeyword{(}\hlargument{ldaFit2}{\ }\hlargument{=}{\ }\hlsymbol{ldaFit2}\hlkeyword{,}{\ }\hlargument{knnFit2}{\ }\hlargument{=}{\ }\hlsymbol{knnFit2}\hlkeyword{,}{\ }\hlargument{svmFit2}{\ }\hlargument{=}{\ }\hlsymbol{svmFit2}\hlkeyword{)}\mbox{}
\normalfont
\end{flushleft}
\end{kframe}
\end{knitrout}


Now, we classify the test observations with each of the six trained models.

\begin{knitrout}
\definecolor{shadecolor}{rgb}{0.969, 0.969, 0.969}\color{fgcolor}\begin{kframe}
\begin{flushleft}
\ttfamily\noindent
\hlsymbol{modelPredictions1}{\ }\hlassignement{\usebox{\hlnormalsizeboxlessthan}-}{\ }\hlfunctioncall{lapply}\hlkeyword{(}\hlfunctioncall{predict}\hlkeyword{(}\hlsymbol{fittedModels1}\hlkeyword{,}{\ }\hlsymbol{testX}\hlkeyword{)}\hlkeyword{,}{\ }\hlsymbol{as.character}\hlkeyword{)}\hspace*{\fill}\\
\hlstd{}\hlsymbol{modelPredictions2}{\ }\hlassignement{\usebox{\hlnormalsizeboxlessthan}-}{\ }\hlfunctioncall{lapply}\hlkeyword{(}\hlfunctioncall{predict}\hlkeyword{(}\hlsymbol{fittedModels2}\hlkeyword{,}{\ }\hlsymbol{testX2}\hlkeyword{)}\hlkeyword{,}{\ }\hlsymbol{as.character}\hlkeyword{)}\hspace*{\fill}\\
\hlstd{}\hlsymbol{modelPredictions}{\ }\hlassignement{\usebox{\hlnormalsizeboxlessthan}-}{\ }\hlfunctioncall{cbind.data.frame}\hlkeyword{(}\hlfunctioncall{do.call}\hlkeyword{(}\hlsymbol{cbind}\hlkeyword{,}{\ }\hlsymbol{modelPredictions1}\hlkeyword{)}\hlkeyword{,}\hspace*{\fill}\\
\hlstd{}{\ }{\ }{\ }{\ }\hlfunctioncall{do.call}\hlkeyword{(}\hlsymbol{cbind}\hlkeyword{,}{\ }\hlsymbol{modelPredictions2}\hlkeyword{)}\hlkeyword{)}\mbox{}
\normalfont
\end{flushleft}
\end{kframe}
\end{knitrout}


At this point, we have each signatures's classifications for the test
observations. We combine the signatures into a {\tt data.frame} to easily export
for usage with the {\tt SQM} package. Each row of {\tt signatures} corresponds to
a single observation in the test data set. Furthermore, notice that each row of
{\tt signatures} has a {\tt signatureID}, {\tt truthClass}, and
{\tt predictedClass}: the {\tt signatureID} identifies the corresponding
signature, the {\tt truthClass} is the true value of the test observation, and
the {\tt predictedClass} is the predicted classification for this test
observation by the current signature specified in {\tt signatureID}. We display
the first few signatures.

\begin{knitrout}
\definecolor{shadecolor}{rgb}{0.969, 0.969, 0.969}\color{fgcolor}\begin{kframe}
\begin{flushleft}
\ttfamily\noindent
\hlsymbol{signatures}{\ }\hlassignement{\usebox{\hlnormalsizeboxlessthan}-}{\ }\hlfunctioncall{cbind}\hlkeyword{(}\hlargument{obsNum}{\ }\hlargument{=}{\ }\hlfunctioncall{seq\usebox{\hlnormalsizeboxunderscore}along}\hlkeyword{(}\hlsymbol{testY}\hlkeyword{)}\hlkeyword{,}{\ }\hlargument{truthClass}{\ }\hlargument{=}{\ }\hlsymbol{testY}\hlkeyword{,}\hspace*{\fill}\\
\hlstd{}{\ }{\ }{\ }{\ }\hlsymbol{modelPredictions}\hlkeyword{)}\hspace*{\fill}\\
\hlstd{}\hlsymbol{signatures}{\ }\hlassignement{\usebox{\hlnormalsizeboxlessthan}-}{\ }\hlfunctioncall{melt}\hlkeyword{(}\hlsymbol{signatures}\hlkeyword{,}{\ }\hlargument{id}{\ }\hlargument{=}{\ }\hlfunctioncall{c}\hlkeyword{(}\hlstring{"{}obsNum"{}}\hlkeyword{,}{\ }\hlstring{"{}truthClass"{}}\hlkeyword{)}\hlkeyword{,}{\ }\hlargument{measure}{\ }\hlargument{=}{\ }\hlfunctioncall{c}\hlkeyword{(}\hlstring{"{}ldaFit1"{}}\hlkeyword{,}\hspace*{\fill}\\
\hlstd{}{\ }{\ }{\ }{\ }\hlstring{"{}knnFit1"{}}\hlkeyword{,}{\ }\hlstring{"{}svmFit1"{}}\hlkeyword{,}{\ }\hlstring{"{}ldaFit2"{}}\hlkeyword{,}{\ }\hlstring{"{}knnFit2"{}}\hlkeyword{,}{\ }\hlstring{"{}svmFit2"{}}\hlkeyword{)}\hlkeyword{)}\hspace*{\fill}\\
\hlstd{}\hlsymbol{signatures}{\ }\hlassignement{\usebox{\hlnormalsizeboxlessthan}-}{\ }\hlfunctioncall{mutate}\hlkeyword{(}\hlsymbol{signatures}\hlkeyword{,}{\ }\hlargument{obsNum}{\ }\hlargument{=}{\ }NULL\hlkeyword{)}\hspace*{\fill}\\
\hlstd{}\hlsymbol{signatures}{\ }\hlassignement{\usebox{\hlnormalsizeboxlessthan}-}{\ }\hlfunctioncall{with}\hlkeyword{(}\hlsymbol{signatures}\hlkeyword{,}{\ }\hlfunctioncall{cbind.data.frame}\hlkeyword{(}\hlargument{signatureID}{\ }\hlargument{=}{\ }\hlsymbol{variable}\hlkeyword{,}\hspace*{\fill}\\
\hlstd{}{\ }{\ }{\ }{\ }\hlsymbol{truthClass}\hlkeyword{,}{\ }\hlargument{predictedClass}{\ }\hlargument{=}{\ }\hlsymbol{value}\hlkeyword{)}\hlkeyword{)}\hspace*{\fill}\\
\hlstd{}\hlfunctioncall{head}\hlkeyword{(}\hlsymbol{signatures}\hlkeyword{)}\mbox{}
\normalfont
\end{flushleft}
\begin{verbatim}
##   signatureID truthClass predictedClass
## 1     ldaFit1  grey.soil      grey.soil
## 2     ldaFit1  grey.soil      grey.soil
## 3     ldaFit1  grey.soil      grey.soil
## 4     ldaFit1  grey.soil      grey.soil
## 5     ldaFit1  grey.soil      grey.soil
## 6     ldaFit1  grey.soil damp.grey.soil
\end{verbatim}
\end{kframe}
\end{knitrout}


At this point, we have the signatures as they would be provided to the {\tt SQM}
package. To continue with our example, we write the signatures to a CSV file.
This CSV file would need to be provided in an actual application of the {\tt SQM}
package.

\begin{knitrout}
\definecolor{shadecolor}{rgb}{0.969, 0.969, 0.969}\color{fgcolor}\begin{kframe}
\begin{flushleft}
\ttfamily\noindent
\hlsymbol{inputFile}{\ }\hlassignement{\usebox{\hlnormalsizeboxlessthan}-}{\ }\hlstring{"{}demo\usebox{\hlnormalsizeboxunderscore}signatures.csv"{}}\hspace*{\fill}\\
\hlstd{}\hlfunctioncall{write.table}\hlkeyword{(}\hlsymbol{signatures}\hlkeyword{,}{\ }\hlargument{file}{\ }\hlargument{=}{\ }\hlsymbol{inputFile}\hlkeyword{,}{\ }\hlargument{sep}{\ }\hlargument{=}{\ }\hlstring{"{},"{}}\hlkeyword{,}{\ }\hlargument{row.names}{\ }\hlargument{=}{\ }\hlnumber{FALSE}\hlkeyword{,}\hspace*{\fill}\\
\hlstd{}{\ }{\ }{\ }{\ }\hlargument{col.names}{\ }\hlargument{=}{\ }\hlnumber{FALSE}\hlkeyword{,}{\ }\hlargument{quote}{\ }\hlargument{=}{\ }\hlnumber{FALSE}\hlkeyword{)}\mbox{}
\normalfont
\end{flushleft}
\end{kframe}
\end{knitrout}


\section{Signature Accuracy}

The CSV file {\tt demo\_signatures.csv} contains the classifications for each
signature. The Accuracy component of the {\tt SQM} package can be employed with
relative ease after the signatures have been constructed. In fact, with our CSV
file in hand, we are able to compute the accuracy measures in one line of R
code. We do that here.

\begin{knitrout}
\definecolor{shadecolor}{rgb}{0.969, 0.969, 0.969}\color{fgcolor}\begin{kframe}
\begin{flushleft}
\ttfamily\noindent
\hlsymbol{outAfAccuracy}{\ }\hlassignement{\usebox{\hlnormalsizeboxlessthan}-}{\ }\hlfunctioncall{afAccuracy}\hlkeyword{(}\hlargument{inputFile}{\ }\hlargument{=}{\ }\hlsymbol{inputFile}\hlkeyword{,}{\ }\hlargument{outputFilePrefix}{\ }\hlargument{=}{\ }\hlstring{"{}demo"{}}\hlkeyword{)}\mbox{}
\normalfont
\end{flushleft}
\end{kframe}
\end{knitrout}


Notice the {\tt outputFilePrefix} argument having the value of {\tt demo}. Two
output files are created that contain the accuracy estimates for the signatures.
The first file generated is {\tt demo\_aggregate.csv}, which is a CSV file that
contains the aggregate accuracy estimates for each signature. The second file
generated is {\tt demo\_byClass.csv}, which is a CSV file that contains the
accuracy measures for each class within each signature. The {\tt aggregate}
measure is useful for quick comparison among signatures, while the class-by-class
breakdown in {\tt byClass} is useful to measure the ability of a signature to
correctly classify an observation for a given class.

The object {\tt outAfAccuracy} contains these same accuracy results in a named
list with two elements: {\tt aggregate} and {\tt byClass}.

First, let's see the first few {\tt aggregate} results:

\begin{knitrout}
\definecolor{shadecolor}{rgb}{0.969, 0.969, 0.969}\color{fgcolor}\begin{kframe}
\begin{verbatim}
##   signatureID accuracy precision recall sensitivity specificity    TPR
## 1     knnFit1   0.9605    0.8814 0.8814      0.8814      0.9763 0.8814
## 2     knnFit2   0.9549    0.8648 0.8648      0.8648      0.9730 0.8648
## 3     ldaFit1   0.9433    0.8298 0.8298      0.8298      0.9660 0.8298
## 4     ldaFit2   0.9381    0.8143 0.8143      0.8143      0.9629 0.8143
## 5     svmFit1   0.9599    0.8798 0.8798      0.8798      0.9760 0.8798
## 6     svmFit2   0.9563    0.8688 0.8688      0.8688      0.9738 0.8688
##      TNR     FPR    FNR Fscore
## 1 0.9763 0.02371 0.1186 0.8814
## 2 0.9730 0.02703 0.1352 0.8648
## 3 0.9660 0.03404 0.1702 0.8298
## 4 0.9629 0.03715 0.1857 0.8143
## 5 0.9760 0.02405 0.1202 0.8798
## 6 0.9738 0.02624 0.1312 0.8688
\end{verbatim}
\end{kframe}
\end{knitrout}


And now, the first few {\tt byClass} results:

\begin{knitrout}
\definecolor{shadecolor}{rgb}{0.969, 0.969, 0.969}\color{fgcolor}\begin{kframe}
\begin{verbatim}
##   signatureID               class accuracy precision recall sensitivity
## 1     knnFit1         cotton.crop   0.9925    0.9589 0.9734      0.9734
## 2     knnFit1      damp.grey.soil   0.9279    0.6290 0.6290      0.6290
## 3     knnFit1           grey.soil   0.9583    0.8828 0.9253      0.9253
## 4     knnFit1            red.soil   0.9863    0.9609 0.9826      0.9826
## 5     knnFit1  vegetation.stubble   0.9666    0.8951 0.7887      0.7887
## 6     knnFit1 very.damp.grey.soil   0.9312    0.8596 0.8444      0.8444
##   specificity    TPR    TNR     FPR     FNR Fscore
## 1      0.9949 0.9734 0.9949 0.00512 0.02657 0.9661
## 2      0.9600 0.6290 0.9600 0.03995 0.37100 0.6290
## 3      0.9672 0.9253 0.9672 0.03284 0.07466 0.9036
## 4      0.9875 0.9826 0.9875 0.01252 0.01741 0.9716
## 5      0.9886 0.7887 0.9886 0.01141 0.21132 0.8385
## 6      0.9578 0.8444 0.9578 0.04224 0.15561 0.8519
\end{verbatim}
\end{kframe}
\end{knitrout}


\end{document}
